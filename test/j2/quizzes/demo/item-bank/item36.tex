\question
What happens in the below code?

\begin{lstlisting}
class A: 
   def __init__(self, i=100): 
      self.i=i 
class B(A): 
   def __init__(self,j=0): 
      self.j=j 
def main(): 
   b= B() 
   print(b.i) 
   print(b.j) 
main() 
\end{lstlisting}

\begin{enumerate}
\item Class \lstinline@B@ inherits all the data fields of class \lstinline@A@.
\item Class \lstinline@B@ needs an Argument.
\item The data field \lstinline@'j'@ cannot be accessed by object \lstinline@b@.
\item Class \lstinline@B@ is inheriting class \lstinline@A@ but the data field \lstinline@'i'@ in \lstinline@A@ cannot be inherited.
\end{enumerate}

